\section{The LZSS Algorithm}
The LZ77 algorithm has a lack of performances due to the constraint on the triplet to send: whenever we have to encode a single symbol, we have to waste $l_{offset} + l_{length}$ bytes because the \textit{offset} and the \textit{length} of the match are both zero. In fact the triplet to send is $[0, 0, \alpha]$. In 1982 Storer and Szymanski \cite{storer1} proposed a different version of the LZ77 algorithm which improves the compression performances through a better management of the size of the dictionary entries.

We can distinguish two kinds of entries: those which contain the information related to an only symbol not find in the \textit{search window} and those which refer to a previous pattern. We can use a flag bit to indicate in which case we are. If the bit is $0$ the entry is made only by the code of a symbol. If the bit is $1$ the entry is made of a pair (\textit{offset}, \textit{length}). This convention allows us to save a lot of space in memory in both cases.

\subsection{LZSS Implementation}
The main difference with the LZ77 implementation lies on the dictionary construction, all the rest of the code is quite the same used in in the version $4$ of the LZ77 exposed in section \ref{subsubsec:pess}, with a pessimistic usage of the function \texttt{strfind()} for pattern matching.

The problem with the usage of flag bits is that every computer works with bytes as fundamental memory unit. Since we cannot simply add a surplus bit to each sequence of bytes forming a dictionary entries, we decided to spend a whole byte to gather the flags for $8$ consecutive entries. The sequence of these $8$ bits is read as a only value and is transmitted as part of the message in front of the entries it is referring to. At the receiver side the knowledge of the number of bytes used to encode \textit{offset}, \textit{length} and \textit{symbol} fields allows a unique interpretation of the message. Once the receiver has build the dictionary, it can proceed with the reconstruction of the original message.