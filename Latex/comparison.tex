\section{Comparison with existing softwares}
In this section we report some results obtained by the comparison between our implementations and other softwares usually installed in common PCs. We use the so called \textit{Canterbury Corpus} as set of files to perform our examination, which is a collection of several types of file whose principal aim is being used to try compression algorithms. I want to mention right now that our implementations have no chance to behave as well as any of the professional software employed by default by a PC. The reasons are mainly two: the first one is that programs that compress files or folders into archives .ZIP use an advanced algorithm, where the simple LZ77 is enhanced by further coding, for example by a Huffman coding (this combined coding is called \textit{LZ77 deflate} and is used by most of the .ZIP compressors). The second reason is about time: since we execute our program inside Matlab, performances are lowered by the infrastructure of the environment and, moreover, \texttt{for} cycles are not very convenient in this particular language. Hence, rather than observing performances in an absolute way, we will investigate the types of files which LZ77 and LZSS are favorable for and we will observe when the implicit dictionary-based coding are a suitable choice or not.
