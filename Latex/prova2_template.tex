\documentclass[11pt, twocolumn]{article}
\usepackage{courier}
\usepackage[T1]{fontenc}
\usepackage[utf8x]{inputenc}
\usepackage{fancyhdr}
\pagestyle{fancy}
\usepackage{color}
\definecolor{document_fontcolor}{rgb}{1, 1, 1}
\color{document_fontcolor}
\usepackage{amsmath}
\usepackage{amssymb}

\makeatletter
\@ifundefined{date}{}{\date{}}
%%%%%%%%%%%%%%%%%%%%%%%%%%%%%% User specified LaTeX commands.

% pacchetto per scrivere in matematichese
\usepackage{listings}% pacchetto per scrivere codice sorgente
% pacchetto per scrivere in italiano
\usepackage{indentfirst}% pacchetto per avere il rientro anche nella prima riga di una sezione
\usepackage{graphicx}% pacchetto per le immagini
\usepackage{subfigure}% pacchetto per le immagini raggruppate
% pacchetto per scrivere gli insiemi matematici
%\usepackage[a4paper, top=1.5cm, bottom=1.5cm, left=1.5cm, right=1.5cm]{geometry}
\usepackage{fancyhdr}% Pacchetto per header persnalizzati
\usepackage{lastpage}% pacchetto per determinare l'ultima pagina, per il footer
\usepackage{extramarks}% Pacchetto per header e footer
\usepackage{lipsum}% Pacchetto per generare un "Lorem Ipsum"
\usepackage{changepage}% pacchetto per impostare i margini della prima pagina
% pacchetto per usare il font Courier per i piccoli pezzi di codice o i nomi dei file
\usepackage{amsfonts}
% pacchetto per equazioni. Usato per splittare lunghe equazioni
\usepackage{amsthm}% pacchetto per i teoremi, da dichiarare DOPO amsmath!
%\usepackage[english]{babel}		% scelgo l'inglese come lingua
\usepackage{verbatim}% pacchetto per i commenti in blocco
\usepackage{listings}% pacchetto per scrivere codice sorgente
% pacchetto per definire nuovi colori
\usepackage{xcolor}% pacchetto per i colori
\usepackage{caption}% pacchetto per le intestazioni nel codice sorgente
\usepackage{appendix}% pacchetto per le appendici
\usepackage{float}	% lo uso per mettere le immagini dove voglio con H

% Margins
\topmargin=-0.45in
\evensidemargin=0in
\oddsidemargin=0in
\textwidth=6.5in
\textheight=9.0in
\headsep=0.25in 

\linespread{1.1}	% spazio tra le linee

%%%%%%%%%%%%%%%%%%%%%%%%%%%%%%%%%%%%%%%%%%%%%%
%	VARIABILI
%%%%%%%%%%%%%%%%%%%%%%%%%%%%%%%%%%%%%%%%%%%%%%

\newcommand{\hmwkTitle}{Homework 5} % Assignment title
\newcommand{\hmwkDueDate}{Venerdì,\ 6 giugno,\ 2014} % Due date
\newcommand{\hmwkClass}{Analisi di Immagini e Video} % Course/class
\newcommand{\hmwkClassTime}{} % Class/lecture time
\newcommand{\hmwkClassInstructor}{Prof. Pietro Zanuttigh} % Teacher/lecturer
\newcommand{\hmwkAuthorName}{Tommaso Martini - 1081580} % Your name

%%%%%%%%%%%%%%%%%%%%%%%%%%%%%%%%%%%%%%%%%%%%%%
%	HEADER e FOOTER
%%%%%%%%%%%%%%%%%%%%%%%%%%%%%%%%%%%%%%%%%%%%%%

\lhead{\hmwkAuthorName} % Top left header
%\chead{\hmwkClass\ (\hmwkClassInstructor\ \hmwkClassTime): \hmwkTitle} % Top center header
\rhead{\firstxmark} % Top right header
\lfoot{\lastxmark} % Bottom left footer
\cfoot{} % Bottom center footer
\rfoot{Pagina\ \thepage\ di\ \pageref{LastPage}} % Bottom right footer
\renewcommand{\headrulewidth}{0.4pt} % Size of the header rule
\renewcommand{\footrulewidth}{0.4pt} % Size of the footer rule

%\setlength\parindent{0pt} % Removes all indentation from paragraphs

%%%%%%%%%%%%%%%%%%%%%%%%%%%%%%%%%%%%%%%%%%%%%%
%	CODICE
%%%%%%%%%%%%%%%%%%%%%%%%%%%%%%%%%%%%%%%%%%%%%%

\definecolor{mygreen}{rgb}{0,0.6,0}
\definecolor{mygray}{rgb}{0.5,0.5,0.5}
\definecolor{mymauve}{rgb}{0.58,0,0.82}

\lstset{ 
  backgroundcolor=,   % choose the background color; you must add \usepackage{color} or \usepackage{xcolor}
  %basicstyle=\footnotesize,        % the size of the fonts that are used for the code
  basicstyle={\small\ttfamily},
  breakatwhitespace=false,         % sets if automatic breaks should only happen at whitespace
  breaklines=true,                 % sets automatic line breaking
 % captionpos=u,                    % sets the caption-position to bottom
  commentstyle=\color{mygreen},    % comment style
  deletekeywords={...},            % if you want to delete keywords from the given language
  escapeinside={\%*}{*)},          % if you want to add LaTeX within your code
  extendedchars=true,              % lets you use non-ASCII characters; for 8-bits encodings only, does not work with UTF-8
 % frame=single,                    % adds a frame around the code
  frame=b,
  keepspaces=true,                 % keeps spaces in text, useful for keeping indentation of code (possibly needs columns=flexible)
  keywordstyle=,       % keyword style
  language=Matlab,                 % the language of the code
  morekeywords={*,...},            % if you want to add more keywords to the set
  numbers=none,                    % where to put the line-numbers; possible values are (none, left, right)
  numbersep=5pt,                   % how far the line-numbers are from the code
  numberstyle=\tiny\color{mygray}, % the style that is used for the line-numbers
  rulecolor=,         % if not set, the frame-color may be changed on line-breaks within not-black text (e.g. comments (green here))
  showspaces=false,                % show spaces everywhere adding particular underscores; it overrides 'showstringspaces'
  showstringspaces=false,          % underline spaces within strings only
  showtabs=false,                  % show tabs within strings adding particular underscores
  stepnumber=1,                    % the step between two line-numbers. If it's 1, each line will be numbered
  stringstyle=\color{mymauve},     % string literal style
  tabsize=4,                       % sets default tabsize to 2 spaces
  title=\lstname,                  % show the filename of files included with \lstinputlisting; also try caption instead of title
  numberbychapter = false
}
\renewcommand{\lstlistingname}{Section}

\DeclareCaptionFont{white}{}
\DeclareCaptionFormat{listing}{\colorbox{lightgray}{\parbox{\textwidth}{#1#2#3}}}
\captionsetup[lstlisting]{format=listing, font={tt, color=white}, labelfont={bf, tt, color=white}}
\renewcommand{\captionfont}{\small\ttfamily}
   
%%%%%%%%%%%%%%%%%%%%%%%%%%%%%%%%%%%%%%%%%%%%%%
%	TITOLO
%%%%%%%%%%%%%%%%%%%%%%%%%%%%%%%%%%%%%%%%%%%%%%

\title{
\vspace{2in}
\textmd{\textbf{\hmwkClass:\ \hmwkTitle}}\\
\normalsize\vspace{0.1in}\small{Consegna:\ \hmwkDueDate}\\
\vspace{0.1in}\large{\textit{\hmwkClassInstructor\ \hmwkClassTime}}
\vspace{3in}
}

\author{\textbf{\hmwkAuthorName}}
 % Insert date here if you want it to appear below your name

%----------------------------------------------------------------------------------------

\makeatother

\begin{document}
\maketitle

\newpage

\section*{Introduzione}
In questo homework ci è richiesto di implementare la tecnica di rilevazione del moto introdotta da Lucas e Kanade nel 1981. Di seguito vengono riportate delle brevi spiegazioni riguardo l'implementazione in Matlab dell'algoritmo; si rimanda ai commenti nel codice per una illustrazione più esauriente.

\section*{Soluzione}
Per prima cosa il video su cui vogliamo operare viene caricato in memoria. Per motivi di semplicità computazionale è possibile scegliere un numero arbitrario di frame da leggere, in modo da non dover processare l'intera sequenza di immagini. Una volta importato il video in una matrice quadri-dimensionale, entriamo nel ciclo principale del programma, che scorre ogni frame; in realtà escludiamo il primo e l'ultimo frame, in modo da non avere problemi di frame mancanti quando, in seguito, avremo bisogno dell'immagine precedente e successiva per eseguire la derivata temporale al frame corrente.

Per semplicità implementativa non lavoreremo con tutte e tre le componenti di colore, ma ci porteremo nello spazio YCbCr ed useremo solo la matrice bidimensionale Y, cioè della luminanza, poiché questa contiene gran parte dell'informazione percepibile dall'occhio umano. Estraiamo la luminanza anche per i frame precedente e successivo. 

Del frame corrente calcoliamo le features più significative usando la funzione \texttt{corner()} di Matlab, che di default implementa l'algoritmo di Harris e calcoliamo il gradiente tramite il comando \texttt{gradient()}, che restituisce una colonna per il gradiente lungo l'asse $X$ e una per il gradiente lungo l'asse $Y$.\\

Ci interessa rilevare il moto solo rispetto ai punti caratteristici dell'immagine, perciò eseguiamo un ulteriore ciclo annidato che prende in considerazione solo i punti classificati come features dal comando \texttt{corner()}, che in uscita dà una colonna di coordinate $(x,y)$. Dato che nella nostra implementazione di Lucas-Kanade dovremo lavorare con una finestra quadrata attorno al pixel considerato, preveniamo possibili problemi ignorando i punti caratteristici troppo vicini al bordo dell'immagine, cioè quei punti che avrebbero una finestra uscente dal frame stesso. Questa, ovviamente, è una semplificazione tanto rude quanto drastica, ma per gli scopi di questo esercizio trascuriamo l'accuratezza e la completezza dell'algoritmo, tralasciando i casi più problematici.

Nel caso in cui sia possibile individuare una finestra intera attorno al pixel corrente, invece, la scorriamo interamente e costruiamo la matrice $A$ e il vettore $\mathbf{b}$, definiti come segue:
\begin{align}
A &= \begin{bmatrix}
{\partial \over \partial x}I(p_1) & {\partial \over \partial y}I(p_1) \\
{\partial \over \partial x}I(p_2) & {\partial \over \partial y}I(p_2) \\
\vdots & \vdots \\
{\partial \over \partial x}I(p_M) & {\partial \over \partial y}I(p_M)
\end{bmatrix} \quad p_1, p_2, \ldots, p_M \in W \\
\mathbf{b} &=  \begin{bmatrix}
{\partial \over \partial t}I(p_1) \\
{\partial \over \partial t}I(p_2) \\
\vdots \\
{\partial \over \partial t}I(p_M) \\
\end{bmatrix} \quad p_1, p_2, \ldots, p_M \in W
\end{align}
dove $I$ è l'immagine e $W$ è la finestra di $M$ pixel attorno al pixel considerato.

Per risolvere il sistema:
\begin{equation}
A \begin{bmatrix}
u \\
v
\end{bmatrix} = \mathbf{b}
\end{equation}
secondo i minimi quadrati, ci serviamo del metodo proposto da Lucas e Kanade e valutiamo, invece, il seguente sistema:
\begin{equation}
A^T A \mathbf{d} = A^T \mathbf{b}
\end{equation}

Come viene richiesto dal compito, a questo punto eseguiamo tre controlli sulla matrice $A^T A$ per individuare casi anomali:
\begin{enumerate}
\item
controlliamo la singolarità della matrice $2 \times 2$ $A^TA$. Per fare ciò non usiamo la funzione \texttt{det()} di Matlab per controllare che il gradiente non sia nullo perché, operando questa per via numerica, può restituire facilmente risultati inconsistenti ed errati. Controlliamo, invece, che il rango della matrice (comando Matlab \texttt{rank()}) non sia minore del numero di righe o colonne, vale a dire $2$. Se il rango è minore di $2$ il programma viene bloccato e viene segnalata l'anomalia all'utente;

\item
controlliamo che gli autovalori non siano troppo piccoli, sintomo di grande sensibilità al rumore: dopo aver calcolato gli autovalori di $A^TA$ tramite la funzione \texttt{eig()}, li riordiniamo in ordine derescente e verifichiamo che il minore dei due, vale a dire $\lambda_2$, non scenda sotto una certa soglia, che possiamo scegliere arbitrariamente, ma che è stata impostata di default a $0.001$; anche in questo caso, qualora si riscontrasse questo problema, il programma verrebbe fermato, comunicando all'utente l'evenienza;

\item
controlliamo che il rapporto tra gli autovalori $\lambda_1 / \lambda_2$ non superi una certa soglia, di default impostata a $100$. Se la soglia viene superata, ancora una volta lo si comunica all'utente e si arresta il programma.
\end{enumerate}

Una volta eseguiti tutti e tre i controlli, con l'istruzione \texttt{d = (A' * A)\textbackslash (A' * b)} viene risolto il sistema e vengono calcolate le componenti dei vettori spostamento stimati del pixel centrale della finestra, vale a dire del pixel "feature" considerato.

\section*{Visualizzazione del moto stimato}
Per visualizzare il moto stimato si usa una semplice tecnica: si crea un'immagine Matlab \texttt{figure}, che viene aggiornata con il frame corrente ad ogni passo del ciclo più esterno, sui frame del video. Per ogni corner valido identificato, di cui si è calcolato il vettore spostamento stimato, si disegna sopra il frame una freccia con origine sul corner e direzione e modulo identificati dal vettore spostamento (funzione \texttt{quiver()}). Per una migliore visualizzazione si è prima moltiplicato per un valore arbitrariamente grande il vettore spostamento, nel nostro caso per $10$, così da poter vedere vettori più grandi e più facilmente apprezzabili.

Ricaricando ad ogni passo del ciclo il frame corrente e ridisegnandovi sopra i vettori individuati si ottiene una sorta di video in cui le frecce si muovono indicando la direzione di spostamento stimata.

\end{document}
