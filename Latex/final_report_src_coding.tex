%\documentclass[a4paper, 12pt, twoside, openright]{book}
\documentclass[11pt, twocolumn]{article}

%%%%%%%%%%%%%%%%%%%%%%%%%%%%%%%%%%%%%%%%%%%%%
%%%%%%%%%%%%  PACKETS  %%%%%%%%%%%%%%%%%%%%%%
%%%%%%%%%%%%%%%%%%%%%%%%%%%%%%%%%%%%%%%%%%%%%
\usepackage{changepage}	% pacchetto per impostare i margini della prima pagina
\usepackage[T1]{fontenc}
\usepackage[utf8]{inputenc}
\usepackage{indentfirst}		% pacchetto per avere il rientro anche nella prima riga di una sezione
\usepackage{courier}		% pacchetto per usare il font Courier per i piccoli pezzi di codice o i nomi dei file
\usepackage{amsfonts}
\usepackage{amsmath}		% pacchetto per equazioni
\usepackage{amsthm}		% pacchetto per i teoremi, da dichiarare DOPO amsmath!
\usepackage{amssymb}		% pacchetto per insiemi matematici

\usepackage[english]{babel}		% scelgo l'inglese come lingua
\usepackage{verbatim}	% pacchetto per i commenti in blocco

\usepackage{listings}	% pacchetto per scrivere codice sorgente
\usepackage{color}		% pacchetto per definire nuovi colori
\usepackage{caption}		% pacchetto per le intestazioni nel codice sorgente
\usepackage{appendix}	% pacchetto per le appendici

\usepackage{color}
\definecolor{document_fontcolor}{rgb}{1, 1, 1}
\color{document_fontcolor}

%%%%%%%%%%%%%%%%%%%%%%%%%%%%%%

\usepackage{graphicx}% pacchetto per le immagini
\usepackage{subfigure}% pacchetto per le immagini raggruppate
%\usepackage[a4paper, top=1.5cm, bottom=1.5cm, left=1.5cm, right=1.5cm]{geometry}
\usepackage{fancyhdr}% Pacchetto per header persnalizzati
\usepackage{lastpage}% pacchetto per determinare l'ultima pagina, per il footer
\usepackage{extramarks}% Pacchetto per header e footer
\usepackage{lipsum}% Pacchetto per generare un "Lorem Ipsum"
\usepackage{float}	% lo uso per mettere le immagini dove voglio con H

%%%%%%%%%%%%%%%%%%%%%%%%%%%%%
\makeatletter
\@ifundefined{date}{}{\date{}}

%%%%%%%%%%%%%%%%%%%%%%%%%%%%%
% Margins
\topmargin=-0.45in
\evensidemargin=0in
\oddsidemargin=0in
\textwidth=6.5in
\textheight=9.0in
\headsep=0.25in 

\linespread{1.1}	% spazio tra le linee

%%%%%%%%%%%%%%%%%%%%%%%%%%%%%%%%%%%%%%%%%%%%%%
%	VARIABILI
%%%%%%%%%%%%%%%%%%%%%%%%%%%%%%%%%%%%%%%%%%%%%%

\newcommand{\hmwkTitle}{Final Project} % Assignment title
\newcommand{\hmwkDueDate}{Tuesday,\ 15-th July,\ 2014} % Due date
\newcommand{\hmwkClass}{Source Coding} % Course/class
\newcommand{\hmwkClassTime}{} % Class/lecture time
\newcommand{\hmwkClassInstructor}{Prof. Giancarlo Calvagno} % Teacher/lecturer
\newcommand{\hmwkAuthorName}{Tommaso Martini - 1081580} % Your name

%%%%%%%%%%%%%%%%%%%%%%%%%%%%%%%%%%%%%%%%%%%%%%
%	HEADER e FOOTER
%%%%%%%%%%%%%%%%%%%%%%%%%%%%%%%%%%%%%%%%%%%%%%
%
%\lhead{\hmwkAuthorName} % Top left header
%%\chead{\hmwkClass\ (\hmwkClassInstructor\ \hmwkClassTime): \hmwkTitle} % Top center header
%\rhead{\firstxmark} % Top right header
%\lfoot{\lastxmark} % Bottom left footer
%\cfoot{} % Bottom center footer
%\rfoot{Pagina\ \thepage\ di\ \pageref{LastPage}} % Bottom right footer
%\renewcommand{\headrulewidth}{0.4pt} % Size of the header rule
%\renewcommand{\footrulewidth}{0.4pt} % Size of the footer rule
%
%%\setlength\parindent{0pt} % Removes all indentation from paragraphs


%%%%%%%%%%%%%%%%%%%%%%%%%%%%%
%%%% grafica prima pagina

\pagestyle{fancy}
\usepackage{graphicx}	% pacchetto avanzato per la grafica
\usepackage{wrapfig}		% pacchetto per circondare figure con testo
\usepackage[usenames,dvipsnames,table]{xcolor} % pacchetto per colori
\definecolor{333}{RGB}{51, 51, 51}
%%%%%%%%%%%%%%%%%%%%%%%%%%%%%

%\newtheorem{theorem}{Theorem}	% definisco l'ambiente per teoremi di amsthm

%%%%%%%%%%%%%%%%%%%%%%%%%%%%%%%%%%%%%%%%%
%definisco la struttura per i pezzi di codice
\definecolor{mygreen}{rgb}{0,0.6,0}
\definecolor{mygray}{rgb}{0.5,0.5,0.5}
\definecolor{mymauve}{rgb}{0.58,0,0.82}

\lstset{ 
  backgroundcolor=\color{white},   % choose the background color; you must add \usepackage{color} or \usepackage{xcolor}
  %basicstyle=\footnotesize,        % the size of the fonts that are used for the code
  basicstyle={\small\ttfamily},
  breakatwhitespace=false,         % sets if automatic breaks should only happen at whitespace
  breaklines=true,                 % sets automatic line breaking
 % captionpos=u,                    % sets the caption-position to bottom
  commentstyle=\color{mygreen},    % comment style
  deletekeywords={...},            % if you want to delete keywords from the given language
  escapeinside={\%*}{*)},          % if you want to add LaTeX within your code
  extendedchars=true,              % lets you use non-ASCII characters; for 8-bits encodings only, does not work with UTF-8
 % frame=single,                    % adds a frame around the code
  frame=b,
  keepspaces=true,                 % keeps spaces in text, useful for keeping indentation of code (possibly needs columns=flexible)
  keywordstyle=\color{blue},       % keyword style
  language=Matlab,                 % the language of the code
  morekeywords={*,...},            % if you want to add more keywords to the set
  numbers=none,                    % where to put the line-numbers; possible values are (none, left, right)
  numbersep=5pt,                   % how far the line-numbers are from the code
  numberstyle=\tiny\color{mygray}, % the style that is used for the line-numbers
  rulecolor=\color{black},         % if not set, the frame-color may be changed on line-breaks within not-black text (e.g. comments (green here))
  showspaces=false,                % show spaces everywhere adding particular underscores; it overrides 'showstringspaces'
  showstringspaces=false,          % underline spaces within strings only
  showtabs=false,                  % show tabs within strings adding particular underscores
  stepnumber=1,                    % the step between two line-numbers. If it's 1, each line will be numbered
  stringstyle=\color{mymauve},     % string literal style
  tabsize=4,                       % sets default tabsize to 2 spaces
  title=\lstname,                  % show the filename of files included with \lstinputlisting; also try caption instead of title
  numberbychapter = false
}
\renewcommand{\lstlistingname}{Section}

\DeclareCaptionFont{white}{\color{white}}
\DeclareCaptionFormat{listing}{\colorbox{lightgray}{\parbox{\textwidth}{#1#2#3}}}
\captionsetup[lstlisting]{format=listing, font={tt, color=white}, labelfont={bf, tt, color=white}}
\renewcommand{\captionfont}{\small\ttfamily}
%%%%%%%%%%%%%%%%%%%%%%%%%%%%%%%%%%%%%%%
%
%\fancypagestyle{plain}{		% definisco lo stile delle pagine (numerazione e titoli)
%	\pagestyle{fancy}
%	\fancyhead{}
%	\fancyfoot{}
%	\renewcommand{\headrulewidth}{0pt} 	% elimino la riga sotto l'intestazione
%	\fancyhead[LE]{\leftmark}
%	\fancyhead[RO]{\rightmark}
%	\fancyfoot[LE, RO]{\thepage}
%}
%
%\pagestyle{plain}

%%%%%%%%%%%%%%%%%%%%%%%%%%%%%%%%%%%%%%%%%%%%%%
%	TITLE
%%%%%%%%%%%%%%%%%%%%%%%%%%%%%%%%%%%%%%%%%%%%%%

\title{
\vspace{2in}
\textmd{\textbf{\hmwkClass:\ \hmwkTitle}}\\
\normalsize\vspace{0.1in}\small{Due to:\ \hmwkDueDate}\\
\vspace{0.1in}\large{\textit{\hmwkClassInstructor\ \hmwkClassTime}}
\vspace{3in}
}

\author{\textbf{\hmwkAuthorName}}
 % Insert date here if you want it to appear below your name

%----------------------------------------------------------------------------------------

\makeatother

\begin{document}

%\maketitle

%\frontmatter
%%%%%%%%%%%%%%%%%%%%%%%%%%%%%%%%%%%%%%%
% First Page
%
%\begin{titlepage}
%\begin{center}
%\vbox to 0pt{\vbox to \textheight{\vfill 
%\includegraphics[width=11.5cm]{loghi_unipd/unipd-light} 
%\vfill}\vss}
%
%\hspace{0.5cm}	
%\begin{adjustwidth}{-2cm}{-2cm}
%\begin{minipage}{.20\textwidth}
%  \includegraphics[height=2.5cm]{loghi_unipd/unipdRed}
%\end{minipage}\begin{minipage}{.90\textwidth}
%  \begin{table}[H]
%  \begin{center}
%  \begin{tabular}{c}
%  \scshape{\Large{ \bfseries{Università degli Studi di Padova}}} \\
%  \hline \\
%  \scshape{\Large{Dipartimento di Ingegneria}} \\
%  \scshape{\Large{dell'Informazione}}
%  \end{tabular}
%  \end{center}
%  \end{table}
%\end{minipage}\begin{minipage}{.20\textwidth}
%  \includegraphics[height=2.0cm]{loghi_unipd/DEIeng}
%  %\includegraphics[height=3.5cm]{DEI}
%\end{minipage}
%\end{adjustwidth} 
%
%\vspace{1cm}
%\Large{Corso di Laurea Triennale in Ingegneria dell'Informazione} \\
%\vspace{1.5cm}
%\scshape{\LARGE{\bfseries{Reed-Solomon Codes}}} \\
%\vspace{0.2cm} \linespread{1} \scshape{\large{\bfseries{(Codici di Reed-Solomon)}}}
%\end{center}
%
%\vfill
%\begin{normalsize}
%\begin{flushleft}
%  \hspace{60pt} \textit{Laureando} \hspace{160pt} \textit{Relatore}\\
%  \vspace{5pt}
%  \hspace{30pt} \large{\textbf{Tommaso Martini}} \hspace{70pt} \large{\textbf{Prof. Tomaso Erseghe}}\\
%\end{flushleft}
%\end{normalsize}
%
%\vfill
%\begin{center}
%25 luglio 2013
%\hspace{-0.2cm}
%\line(1, 0){360}
%
%\textsc{Anno Accademico 2012/2013}
%\end{center}
%%%\end{titlepage}
%\twocolumn

%%%%%%%%%%%%%%%%%%%%%%%%%%%%%%%%%%%%%%%%%%%%%%%%
%DEDICA
%\cleardoublepage % make left page blank
%\null
%\vspace{7cm}
%\begin{flushright}
%\textit{To my parents}
%\end{flushright}



%%%%%%%%%%%%%%%%%%%%%%%%%%%%%%%%%%%%%%%%%%%%
%ABSTRACT
%\newenvironment{abstract}{\cleardoublepage\thispagestyle{empty}\null\vfill\begin{center}
%\bfseries\abstractname\end{center}}
%{\vfill\null}
       
%\begin{abstract}
%
%This bachelor degree thesis has the purpose of presenting a general overview on \emph{Reed-Solomon codes} as a subclass of \emph{cyclic codes} and \emph{BCH codes}, using a mathematical approach to describe and prove their many practical aspects. After briefly exposing $Galois Fields$ theory, fundamental for the creation of error correcting codes, we will introduce a systematic encoding strategy through generator polynomial and a decoder based on \emph{Berlekamp-Massey} and \emph{Forney algorithms}. Every explanation will be followed by examples referring to a $RS(255, 223)$ code. Moreover a \textit{Matlab} implementation af a system \textit{encoder - channel - decoder} has been realised and the whole code can be found in this work. \\
%\\
%
%\textit{Questa tesi di laurea triennale ha lo scopo di fornire una panoramica generale sui codici di Reed-Solomon come sottoclasse dei codici ciclici e dei codici BCH, utilizzando un approccio matematico per descrivere e dimostrare i loro numerosi aspetti vantaggiosi. Dopo aver brevemente esposto la teoria dei Campi di Galois, fondamentali per la costruzione di codici correttori d'errore, presenteremo una strategia di codifica sistematica tramite polinomio generatore e un decodificatore basato sugli algoritmi di Berlekamp-Massey e di Forney. Ogni spiegazione sarà seguita da esempi che fanno riferimento ad un codice $RS(255, 223)$. $\grave{E}$ stata, inoltre, realizzata un'implementazione in codice Matlab di un sistema codificatore - canale - decodificatore e l'intero codice può essere trovato all'interno di questo elaborato.}
%\end{abstract}
%
%\tableofcontents

%\mainmatter

%\input{ch1}

%\onecolumn
%\twocolumn

\section{Introduction}
This project has the aim of developing a dictionary-based compression coding technique and comparing the gained performances with those of softwares present in a typical PC. In particular, we are required to develop our own version of the LZ77 and LZSS algorithms.

In the following we will give a brief overview of such algorithms and we will explain how our versions have been implemented, focusing on the implementation choices. Eventually, we will provide a comparison between the two algorithms and their commercial versions, when they work to compress several kinds of file.

\subsection{Dictionary-based coding} \label{subsec:dict-base}
A \textit{dictionary-based coding} is a coding technique which processes a file as a sequence of symbols to build a dictionary, that is a sequence of pairs (key, value), where the value can be either a symbol or a sequence itself. The dictionary, properly formatted, is the message to send to the receiver. The latter has to build back the original sequence exploiting the information contained in the dictionary entries. Actually, for the dictionaries we will be using, we do not care very much about the keys, since the entries are stacked with the same order the receiver will use at its side, therefore we will not talk about a key for a dictionary entry, rather than about a position.

There are two kinds of dictionary-based coding technique, depending on the dictionary form. The first is the \textit{implicit dictionary coding}, which is the case of the LZ77 algorithm, in which the entries are generated and read sequentially, without the need of specifying a key for them. On the other hand, the \textit{explicit dictionary coding} creates some entries which are referred to with their keys and, in the decoding process, are used also non-sequentially. This is the case of the LZ78 algorithm, published by the same authors of the LZ77 one year later.





\section{The LZ77 Algorithm} \label{sec:lz77}
In this section we provide a short overview of the dictionary-based compression algorithm presented by Abrham Lempel and Jacob Ziv in 1977 \cite{ziv1}. The aim of this algorithm is trying to reduce the redundancy due to similar sequence repeating along the message. What it does is basically checking if a certain piece of the sequence has already been found in the past and, if it is the case, we code the whole piece with a reference to the previous alike sequence.

\subsection{The algorithm} \label{subsec:lz77alg}
The algorithm works with two adjacent windows shifting on the right during the coding of the message. The first window is the \textit{searching window} and has length $L_s$; the second one is the \textit{coding window} and has length $L_c$. We want to find a match between any prefix of the \textit{coding window} and a sequence contained in the overall window given by the juxtapposition of the \textit{searching window} and the \textit{coding window}, i.e. it is sufficient that only the first symbol of the match belongs to the \textit{searching window}, while the matched pattern can stretch in through the \textit{coding window} itself. The window lengths $L_s$ and $L_c$ are two parameters of the algorithm and their choice strongly influences the compression performances, we will see how in the last part of this report.
 
Once we have found a match between a prefix of the \textit{coding window} and another piece of message, we code it as a triplet (\textit{offset}, \textit{length}, \textit{symbol}) where:
\begin{itemize}
\item
\textbf{offset} denotes the number of back hops we have to do, starting from the first symbol of the \textit{coding sequence}, to find the beginning of the matched string; it is clear that, for how it is defined, the \textit{offset} cannot exceeds the \textit{searching window} dimension;

\item
\textbf{length} is the length of the matched string or, equivalently, of the prefix of the \textit{coding window} which has been matched. This value cannot exceed $L_c$, since this is the maximum dimension of the pattern we are looking for;

\item
\textbf{symbol} is the first symbol after the matched prefix of the \textit{coding window}. Its encoding assures the functioning of the algorithm also in case of no matches. 
\end{itemize}

After the generation and the storage of a triplet inside the dictionary, we shift the windows such that the first symbol of the \textit{coding window} corresponds to the first non encoded symbol.

\subsection{LZ77 Implementation} \label{subsec:lz77implem}
In the following paragraphs we will present our own implementation in Matlab of the algorithm, spanning the different versions we produced. As a matter of fact, several editions of the same program have been written, adopting different approaches in some parts of the code. Actually the versions differ above all for the pattern matching that has been adopted, while the rest of the code remains basically the same.

\subsubsection{Basic Coding Algorithm} \label{subsubsec:basiclz77}
First of all we choose a file and write it as a sequence of bytes (\texttt{uint8} in Matlab). We assume this is the original message, whose alphabet is made of all the possible combinations of $8$ bit and has size $M = 2^8 = 256$. The first symbol of the sequence is encoded as a single symbol directly, because placing the \textit{coding window} in the first positions makes no sense, since we would have no \textit{searching window} to exploit. Then, the \textit{coding window} starts from the second position of the sequence and, until there are at least $L_s$ symbols before it, the \textit{searching window} is shrunk to fit the available positions. The same thing happen to the \textit{coding window} when we are reaching the end of the file: if less than $L_c$ symbols are left, we use a shorter \textit{coding window}. Note that, in order to have always a symbol to insert in the \textit{symbol} field of the last triplet, near the end of the file the \textit{coding window} is shrunk such that the last symbol of the stream is left out.

Once the dictionary has been created, we want to write it using the least amount of memory we can. Since the values of $L_s$ and $L_c$ are fixed and determine the maximum values for the fields \textit{offset} and \textit{length} and we know that a symbol is represented with a single byte, we can estimate a maximum number of bytes to use for each triplet as:
\begin{align}
l_{offset} &= \left \lceil \frac{\lceil log_2L_s \rceil}{8} \right \rceil \text{bytes} \\
l_{length} &= \left \lceil \frac{\lceil log_2(L_s + L_c - 1) \rceil }{8} \right \rceil \text{bytes} 
\end{align}

We use exactly $l_{offset} + l_{length} + 1$ bytes for each triplet and encode separately its three components. Then we create the message to be sent by concatenating the bytes related to each entry of the dictionary, starting from the first one. Since we use a constant number of bytes for each entry and the receiver knows how many bytes are employed for each triplet component, it is easy to build the dictionary at the receiver side and perform the decoding to restore the original file.

\subsubsection{Version 1: Basic}
As previously mentioned, the several versions of the algorithms differs by their pattern matching technique to find The first, basic, version of the program executes the pattern matching by brute force: we compare the first character of the \textit{coding window} with each element of the \textit{search window}. When a match occurs, we compare the second element of the \textit{coding window} with the following symbol of the \textit{search window} and so on until we find two different symbols. Then, if the length of the match is longer than the longest match found so far, we replace it and save also the starting point of the match in the \textit{search window}. If, in the end, the longest match has zero length, this means that no match has been found and we have to encode one only symbol. This basic algorithm is very wasteful, because we have to try every possible position.

\subsubsection{Version 2: KMP}
A first improvement has been brought by using the KMP (Knuth-Morris-Pratt) pattern mathcing algorithm \cite{knuth1}. This algorithm is optimum for the research of a pattern inside a string: it takes a time $\mathcal{O}(n + k)$, where $n$ is the length of the pattern and $k$ is the length of the string. We though about using the KMP algorithm where the pattern $\mathbf{c}$ is the \textit{coding window} and the overall string $[\mathbf{s}, \mathbf{c}]$ is the \textit{search window} concatenated with the \textit{coding window}.

The KMP algorithm starts with the brute force approach, comparing the first character of the pattern with each character of the string, but it allows to improve performances in pattern matching because, thanks to previous failed matches, it realize when it is possible to jump ahead and discard some trials, which will surely fail. We used the "jump mechanism" of KMP to realize a second version of the program, but the time performances keep remaining very poor, due to the many nested cycle we have to use, which are not very fast to be executed in Matlab. Moreover, each different pattern needs a \textit{matching function} to be computed, which increases the number of necessary computations.

\subsubsection{Version 3: optimistic \texttt{strfind()}}
The third version uses the Matlab function \texttt{strfind()} for pattern matching, which is optimized to work in Matlab. We simply invoke this method using the \textit{search window} as string and the \textit{coding window} as pattern to be found. If no matches occurs we discard the last symbol of the \textit{coding window} and repeat the procedure until either a match is found or we reduce the length of the pattern to zero, that means that we have to encode the symbol individually.

The term \textit{optimistic} refers to the order we take different pattern: we are confident to find the longest match, and then the best one, quite soon and therefore we start by looking for the entire pattern. Of course, if there is little redundancy and there is often no match found, we have to repeat the cycle $L_c$ times for each different \textit{coding window}.

\subsubsection{Version 4: pessimistic \texttt{strfind()}} \label{subsubsec:pess}
From an empirical point of view we found that, often, we have not many long matches and hence the \textit{optimistic} approach is not very suitable. The \textit{pessimistic} approach consists in starting to look for a match with the shortest possible prefix of the \textit{coding window} and executing the pattern matching through \texttt{strfind()} until we find no more matches. In this case we consider the previous match, which is the longest one, to build the current triplet. If there are no matches we know it from the first iteration and this allows us to save a lot of time.

This fourth version results to be the faster one; anyway, we must be aware that performances are stringly influenced by what kind of file we are trying to compress. Our experiments involved most of all text files or human-written files, which do not present a large redundancy. On one hand this makes the \textit{pessimistic} approach the best one in terms of time performances; on the other hand, the LZ77 algorithm is not the best choice for compression for files with low autocorrelation.

All the results presented in the following has been obtained with this version of the program.

\section{The LZSS Algorithm}
The LZ77 algorithm has a lack of performances due to the constraint on the triplet size: whenever we encode a single symbol, we waste $l_{offset} + l_{length}$ bits because the \textit{offset} and the \textit{length} of the match are both zero, in fact the triplet to send is $[0, 0, \alpha]$. In 1982 Storer and Szymanski \cite{storer1} proposed a different version of the LZ77 algorithm which improves the compression performances through a better management of the size of the dictionary entries.

We can distinguish two kinds of entries: those which contain the information related to an only symbol not find in the \textit{searching window} and those which refer to a previous pattern. We can use a flag bit to indicate in which case we are. If the bit is $0$ the entry is made only by the code of a symbol. If the bit is $1$ the entry is made of a pair (\textit{offset}, \textit{length}). This convention allows us to save a lot of space in both cases.

The encoding of a single symbol implies the employment of $9$ bits: $8$ for the symbol itself and one for the $0$ flag. The encoding of a pair, instead, takes $l_{offset} + l_{length} + 1$ bits. It is clear that it could be more convenient to encode some symbols independently even if a match has been found, if the match length is shorter than:
\begin{equation}
k = \left \lceil \frac{l_{offset} + l_{length} + 1}{9} \right \rceil
\end{equation}

\subsection{LZSS Implementation}
The main difference with the LZ77 implementation lies on the dictionary construction, all the rest of the code is quite the same used in the implementation of the LZ77 exposed in section \ref{subsubsec:pess}, with a pessimistic usage of the function \texttt{strfind()} for pattern matching.

Once again, to save memory, we first look at the dictionary as a stream of bits lining up its rows, then we write it as a sequence of bytes and, eventually, we attach in front of it three bytes expressing the number of bits employed to encode the fields \textit{offset}, \textit{length} and \textit{symbol}.

\section{Comparison with existing softwares}
In this section we report some results obtained by the comparison between our implementations and other softwares usually installed in common PCs. We use the so called \textit{Canterbury Corpus} as set of files to perform our examination, which is a collection of several types of file whose principal aim is being used to try compression algorithms. I want to mention right now that our implementations have no chance to behave as well as any of the professional software employed by default by a PC. The reasons are mainly two: the first one is that programs that compress files or folders into archives .ZIP use an advanced algorithm, where the simple LZ77 is enhanced by further coding, for example by a Huffman coding (this combined coding is called \textit{LZ77 deflate} and is used by most of the .ZIP compressors). The second reason is about time: since we execute our program inside Matlab, performances are lowered by the infrastructure of the environment and, moreover, \texttt{for} cycles are not very convenient in this particular language. Hence, rather than observing performances in an absolute way, we will investigate the types of files which LZ77 and LZSS are favorable for and we will observe when the implicit dictionary-based coding are a suitable choice or not.


\begin{thebibliography}{100}

	\bibitem{ziv1} J. ZIV and A. LEMPEL {\em A Universal Algorithm for Sequential Data Compression}, 1977, IEEE Transactions on Information Theory, vol.23.
	
	\bibitem{knuth1} D. KNUTH, J.H. MORRIS and V. PRATT {\em Fast Pattern Matching in Strings}, 1977, SIAM Journal on Computing, vol.6.
	
	\bibitem{storer1} J.A. STORER and T.G. SZYMANSKI {\em Data Compression via Textual Substitution}, 1982, J.ACM, vol.29.
	
%	\bibitem{ziv1} FASULLLOOOOOOOOO {\em A Universal Algorithm for Sequential Data Compression}, 1977, IEEE Transactions on Information Theory, vol.23.
%
%	\bibitem{berlekamp1} E.R. BERLEKAMP {\em Algebraic Coding Theory}, 1968, McGraw-Hill, New York (USA).
%
%	\bibitem{birkhoff1} G. BIRKHOFF and S. Mac LANE {\em A Survey of Modern Algebra}, 1953, Macmillan, New York, (USA).	
%	
%	\bibitem{blahut1} R.E. BLAHUT {\em Theory and Practice of Error Control Codes}, 1983, Addison-Wesley Publishing Company.
%
%	\bibitem{bose1} R.C. BOSE and D.K. RAY-CHAUDHURI {\em On a Class of Error-Correcting Binary Group Codes}, 1960, Inform. Contr., vol.3, pp.~68-79.
%	
%	\bibitem{clark1} G.C. CLARK JR. and J. BIBB CAIN {\em Error-Correction Coding for Digital Communications}, 1981, Plenum Press, New York (USA).
%
%	\bibitem{gallager1} R.G. GALLAGER {\em Information Theory and Reliable Communication}, 1968, Wiley, USA.	
%	
%	\bibitem{garding1} L. G\AA RDING  and T. TAMBOUR {\em Algebra for Computer Science}, 1988, Springer-Verlag, Berlin (GER).
%
%	\bibitem{hamming1} R.W. HAMMING {\em Error Detecting and Error Correcting Codes}, 1950, Bell System Technical Journal, vol.29, pp.~147-150.
%
%	\bibitem{hartley1} R.V.L. HARTLEY {\em Transmission of Information}, 1928, Bell System Technical Journal, vol.7, num.3.	
%
%	\bibitem{hocquenghem1} A. HOCQUENGHEM {\em Codes Correcteurs d'Erreurs}, 1959, Chiffres, vol.2, pp.~147-156.	
%
%	\bibitem{ihara1} S. IHARA {\em Information Theory for Continuous Systems}, 1993, World Scientific Publishing.
%
%	\bibitem{lin1} S. LIN and D.J. COSTELLO JR. {\em Error Control Coding: Fundamentals and Applications}, 1983, Prentice-Hall, Englewood Cliffs, New Jersey (USA).
%	
%	\bibitem{vanlint1} J.H. van LINT {\em Introduction to Coding Theory}, 1982, Springer, Berlin (GER).
%
%	\bibitem{massey1} J.L. MASSEY {\em Shift-Register Synthesis and BCH Decoding}, 1969, IEEE Trans. Inform. Theory, vol.IT-15, num.1, pp.~122-127.
%
%	\bibitem{monti1} C.M. MONTI {\em Teoria dei Codici: Codici a Blocco}, 1995, Libreria Progetto, Padova (ITA).
%
%	\bibitem{nyquist1} H. NYQUIST {\em Certain Topics in Telegraph Transmission Theory}, 1928, Transactions of the American Institute of Electrical Engineers, vol.47.
%
%	\bibitem{peterson1} W.W. PETERSON {\em Encoding and Error-Correction Procedures for the Bose-Chaudhuri Codes}, 1960, IRE Trans. Inform. Theory, vol.IT-6.
%
%	\bibitem{pierce1} J.R. PIERCE {\em An Introduction to Information Theory. Symbols, Signals and Noise}, 1980, Dover Publications, New York (USA).
%		
%	\bibitem{prange1} E. PRANGE {\em Cyclic Error Correcting Codes in two Symbols}, 1957, Tech. Note AFCRC-TN-57-103, Air Force Cambridge Research Center, Cambridge, MA (USA).
%	
%	\bibitem{reed1} I.S. REED and G. SOLOMON {\em Polynomial Codes over Certain Finite Fields}, 1960, Journal of the Society for Industrial and Applied Mathematics, SIAM, vol.8, pp.~300-304.
%	
%	\bibitem{richardson1} T. RICHARDSON and R. URBANKE {\em Modern Coding Theory}, 2008, Cambridge University Press, New York (USA).	
%	
%	\bibitem{shannon1} C.E. SHANNON {\em A Mathematical Theory of Communication}, 1948, Bell System Technical Journal, vol.27, pp.379-423.
%	
%	\bibitem{sweeney1} P. SWEENEY {\em Error Control Coding}, 2002, Wiley, Chichester, West Sussex (ENG).
%	
%	\bibitem{waerden1} B.L. van der WAERDEN {\em Modern Algebra}, 1950, Frederick Ungar, New York (USA).
%
%	\bibitem{wicker1} S.B. WICKER and V.K. BHARGAVA {\em Reed-Solomon Codes and their Applications}, 1994, IEEE Press, New York (USA).
	
\end{thebibliography}

\end{document}