\documentclass[11pt, twocolumn]{article}

\usepackage[T1]{fontenc}
\usepackage[utf8]{inputenc}
\usepackage{indentfirst}		% pacchetto per avere il rientro anche nella prima riga di una sezione
\usepackage{courier}		% pacchetto per usare il font Courier per i piccoli pezzi di codice o i nomi dei file
\usepackage{amsfonts}
\usepackage{amsmath}		% pacchetto per equazioni
\usepackage{amsthm}		% pacchetto per i teoremi, da dichiarare DOPO amsmath!
\usepackage{amssymb}		% pacchetto per insiemi matematici
\usepackage[english]{babel}		% scelgo l'inglese come lingua
\usepackage{verbatim}	% pacchetto per i commenti in blocco
\usepackage{graphicx}% pacchetto per le immagini
\usepackage{subfigure}% pacchetto per le immagini raggruppate
\usepackage{fancyhdr}% Pacchetto per header personalizzati
\usepackage{float}	% lo uso per mettere le immagini dove voglio con H

\topmargin=-0.45in
\evensidemargin=0in
\oddsidemargin=0in
\textwidth=6.5in
\textheight=9.0in
\headsep=0.25in 
\linespread{1.1}	% spazio tra le linee


\newcommand{\hmwkAuthorName}{Tommaso Martini - 1081580}
\lhead{LZ77 and LZSS Algorithms} % Top left header
\rhead{\hmwkAuthorName} % Top right header

\begin{document}

\title{{\Large Source Coding - Final Project} \\ \textsc{\LARGE \textbf{The LZ77 and LZSS Algorithms for Data Compression}}}
\date{July 25, 2014}
\author{Tommaso Martini (108 15 80)}
        
\maketitle

\section{Introduction}
This project has the aim of developing a dictionary-based compression coding technique and comparing the gained performances with those of softwares present in a typical PC. In particular, we are required to develop our own version of the LZ77 and LZSS algorithms.

In the following we will give a brief overview of such algorithms and we will explain how our versions have been implemented, focusing on the implementation choices. Eventually, we will provide a comparison between the two algorithms and their commercial versions, when they work to compress several kinds of file.

\subsection{Dictionary-based coding} \label{subsec:dict-base}
A \textit{dictionary-based coding} is a coding technique which processes a file as a sequence of symbols to build a dictionary, that is a sequence of pairs (key, value), where the value can be either a symbol or a sequence itself. The dictionary, properly formatted, is the message to send to the receiver. The latter has to build back the original sequence exploiting the information contained in the dictionary entries. Actually, for the dictionaries we will be using, we do not care very much about the keys, since the entries are stacked with the same order the receiver will use at its side, therefore we will not talk about a key for a dictionary entry, rather than about a position.

There are two kinds of dictionary-based coding technique, depending on the dictionary form. The first is the \textit{implicit dictionary coding}, which is the case of the LZ77 algorithm, in which the entries are generated and read sequentially, without the need of specifying a key for them. On the other hand, the \textit{explicit dictionary coding} creates some entries which are referred to with their keys and, in the decoding process, are used also non-sequentially. This is the case of the LZ78 algorithm, published by the same authors of the LZ77 one year later.





\section{The LZ77 Algorithm} \label{sec:lz77}
In this section we provide a short overview of the dictionary-based compression algorithm presented by Abrham Lempel and Jacob Ziv in 1977 \cite{ziv1}. The aim of this algorithm is trying to reduce the redundancy due to similar sequence repeating along the message. What it does is basically checking if a certain piece of the sequence has already been found in the past and, if it is the case, we code the whole piece with a reference to the previous alike sequence.

\subsection{The algorithm} \label{subsec:lz77alg}
The algorithm works with two adjacent windows shifting on the right during the coding of the message. The first window is the \textit{searching window} and has length $L_s$; the second one is the \textit{coding window} and has length $L_c$. We want to find a match between any prefix of the \textit{coding window} and a sequence contained in the overall window given by the juxtapposition of the \textit{searching window} and the \textit{coding window}, i.e. it is sufficient that only the first symbol of the match belongs to the \textit{searching window}, while the matched pattern can stretch in through the \textit{coding window} itself. The window lengths $L_s$ and $L_c$ are two parameters of the algorithm and their choice strongly influences the compression performances, we will see how in the last part of this report.
 
Once we have found a match between a prefix of the \textit{coding window} and another piece of message, we code it as a triplet (\textit{offset}, \textit{length}, \textit{symbol}) where:
\begin{itemize}
\item
\textbf{offset} denotes the number of back hops we have to do, starting from the first symbol of the \textit{coding sequence}, to find the beginning of the matched string; it is clear that, for how it is defined, the \textit{offset} cannot exceeds the \textit{searching window} dimension;

\item
\textbf{length} is the length of the matched string or, equivalently, of the prefix of the \textit{coding window} which has been matched. This value cannot exceed $L_c$, since this is the maximum dimension of the pattern we are looking for;

\item
\textbf{symbol} is the first symbol after the matched prefix of the \textit{coding window}. Its encoding assures the functioning of the algorithm also in case of no matches. 
\end{itemize}

After the generation and the storage of a triplet inside the dictionary, we shift the windows such that the first symbol of the \textit{coding window} corresponds to the first non encoded symbol.

\subsection{LZ77 Implementation} \label{subsec:lz77implem}
In the following paragraphs we will present our own implementation in Matlab of the algorithm, spanning the different versions we produced. As a matter of fact, several editions of the same program have been written, adopting different approaches in some parts of the code. Actually the versions differ above all for the pattern matching that has been adopted, while the rest of the code remains basically the same.

\subsubsection{Basic Coding Algorithm} \label{subsubsec:basiclz77}
First of all we choose a file and write it as a sequence of bytes (\texttt{uint8} in Matlab). We assume this is the original message, whose alphabet is made of all the possible combinations of $8$ bit and has size $M = 2^8 = 256$. The first symbol of the sequence is encoded as a single symbol directly, because placing the \textit{coding window} in the first positions makes no sense, since we would have no \textit{searching window} to exploit. Then, the \textit{coding window} starts from the second position of the sequence and, until there are at least $L_s$ symbols before it, the \textit{searching window} is shrunk to fit the available positions. The same thing happen to the \textit{coding window} when we are reaching the end of the file: if less than $L_c$ symbols are left, we use a shorter \textit{coding window}. Note that, in order to have always a symbol to insert in the \textit{symbol} field of the last triplet, near the end of the file the \textit{coding window} is shrunk such that the last symbol of the stream is left out.

Once the dictionary has been created, we want to write it using the least amount of memory we can. Since the values of $L_s$ and $L_c$ are fixed and determine the maximum values for the fields \textit{offset} and \textit{length} and we know that a symbol is represented with a single byte, we can estimate a maximum number of bytes to use for each triplet as:
\begin{align}
l_{offset} &= \left \lceil \frac{\lceil log_2L_s \rceil}{8} \right \rceil \text{bytes} \\
l_{length} &= \left \lceil \frac{\lceil log_2(L_s + L_c - 1) \rceil }{8} \right \rceil \text{bytes} 
\end{align}

We use exactly $l_{offset} + l_{length} + 1$ bytes for each triplet and encode separately its three components. Then we create the message to be sent by concatenating the bytes related to each entry of the dictionary, starting from the first one. Since we use a constant number of bytes for each entry and the receiver knows how many bytes are employed for each triplet component, it is easy to build the dictionary at the receiver side and perform the decoding to restore the original file.

\subsubsection{Version 1: Basic}
As previously mentioned, the several versions of the algorithms differs by their pattern matching technique to find The first, basic, version of the program executes the pattern matching by brute force: we compare the first character of the \textit{coding window} with each element of the \textit{search window}. When a match occurs, we compare the second element of the \textit{coding window} with the following symbol of the \textit{search window} and so on until we find two different symbols. Then, if the length of the match is longer than the longest match found so far, we replace it and save also the starting point of the match in the \textit{search window}. If, in the end, the longest match has zero length, this means that no match has been found and we have to encode one only symbol. This basic algorithm is very wasteful, because we have to try every possible position.

\subsubsection{Version 2: KMP}
A first improvement has been brought by using the KMP (Knuth-Morris-Pratt) pattern mathcing algorithm \cite{knuth1}. This algorithm is optimum for the research of a pattern inside a string: it takes a time $\mathcal{O}(n + k)$, where $n$ is the length of the pattern and $k$ is the length of the string. We though about using the KMP algorithm where the pattern $\mathbf{c}$ is the \textit{coding window} and the overall string $[\mathbf{s}, \mathbf{c}]$ is the \textit{search window} concatenated with the \textit{coding window}.

The KMP algorithm starts with the brute force approach, comparing the first character of the pattern with each character of the string, but it allows to improve performances in pattern matching because, thanks to previous failed matches, it realize when it is possible to jump ahead and discard some trials, which will surely fail. We used the "jump mechanism" of KMP to realize a second version of the program, but the time performances keep remaining very poor, due to the many nested cycle we have to use, which are not very fast to be executed in Matlab. Moreover, each different pattern needs a \textit{matching function} to be computed, which increases the number of necessary computations.

\subsubsection{Version 3: optimistic \texttt{strfind()}}
The third version uses the Matlab function \texttt{strfind()} for pattern matching, which is optimized to work in Matlab. We simply invoke this method using the \textit{search window} as string and the \textit{coding window} as pattern to be found. If no matches occurs we discard the last symbol of the \textit{coding window} and repeat the procedure until either a match is found or we reduce the length of the pattern to zero, that means that we have to encode the symbol individually.

The term \textit{optimistic} refers to the order we take different pattern: we are confident to find the longest match, and then the best one, quite soon and therefore we start by looking for the entire pattern. Of course, if there is little redundancy and there is often no match found, we have to repeat the cycle $L_c$ times for each different \textit{coding window}.

\subsubsection{Version 4: pessimistic \texttt{strfind()}} \label{subsubsec:pess}
From an empirical point of view we found that, often, we have not many long matches and hence the \textit{optimistic} approach is not very suitable. The \textit{pessimistic} approach consists in starting to look for a match with the shortest possible prefix of the \textit{coding window} and executing the pattern matching through \texttt{strfind()} until we find no more matches. In this case we consider the previous match, which is the longest one, to build the current triplet. If there are no matches we know it from the first iteration and this allows us to save a lot of time.

This fourth version results to be the faster one; anyway, we must be aware that performances are stringly influenced by what kind of file we are trying to compress. Our experiments involved most of all text files or human-written files, which do not present a large redundancy. On one hand this makes the \textit{pessimistic} approach the best one in terms of time performances; on the other hand, the LZ77 algorithm is not the best choice for compression for files with low autocorrelation.

All the results presented in the following has been obtained with this version of the program.

\section{The LZSS Algorithm}
The LZ77 algorithm has a lack of performances due to the constraint on the triplet size: whenever we encode a single symbol, we waste $l_{offset} + l_{length}$ bits because the \textit{offset} and the \textit{length} of the match are both zero, in fact the triplet to send is $[0, 0, \alpha]$. In 1982 Storer and Szymanski \cite{storer1} proposed a different version of the LZ77 algorithm which improves the compression performances through a better management of the size of the dictionary entries.

We can distinguish two kinds of entries: those which contain the information related to an only symbol not find in the \textit{searching window} and those which refer to a previous pattern. We can use a flag bit to indicate in which case we are. If the bit is $0$ the entry is made only by the code of a symbol. If the bit is $1$ the entry is made of a pair (\textit{offset}, \textit{length}). This convention allows us to save a lot of space in both cases.

The encoding of a single symbol implies the employment of $9$ bits: $8$ for the symbol itself and one for the $0$ flag. The encoding of a pair, instead, takes $l_{offset} + l_{length} + 1$ bits. It is clear that it could be more convenient to encode some symbols independently even if a match has been found, if the match length is shorter than:
\begin{equation}
k = \left \lceil \frac{l_{offset} + l_{length} + 1}{9} \right \rceil
\end{equation}

\subsection{LZSS Implementation}
The main difference with the LZ77 implementation lies on the dictionary construction, all the rest of the code is quite the same used in the implementation of the LZ77 exposed in section \ref{subsubsec:pess}, with a pessimistic usage of the function \texttt{strfind()} for pattern matching.

Once again, to save memory, we first look at the dictionary as a stream of bits lining up its rows, then we write it as a sequence of bytes and, eventually, we attach in front of it three bytes expressing the number of bits employed to encode the fields \textit{offset}, \textit{length} and \textit{symbol}.

\section{Performances of the Algorithms}
We want now to focus our attention on the performances reached by our implementations of the LZ77 and LZSS algorithms in relation both with the type of file we are trying to compress and the parameters setting, that is the lengths of the \textit{searching window} and the \textit{coding window}.

\subsection{Testing files}
All the experiments we performed have been executed on seven specific types of files, part of them taken from the \textit{Canterbury Corpus}, a collection of files with several formats to be used as sample files for compression testing. The used files are here reported:

\begin{itemize}
\item
\textbf{\texttt{rep.txt}}: simple text file containing a sequence of character \texttt{A} repeated $10000$ times ($10000$ bytes);

\item
\textbf{\texttt{per}}: file containing $10$ repetitions of a random sequence of $1000$ bytes ($10000$ bytes);

\item
\textbf{\texttt{ran}}: file generated by concatening $10000$ random bytes ($10000$ bytes);

\item
\textbf{\texttt{shak.txt}}: simple text file reporting a piece of a Shakespeare poem ($13741$ bytes);

\item
\textbf{\texttt{web.html}}: html format page ($24603$ bytes);

\item
\textbf{\texttt{code.c}}: piece of \texttt{c} code ($11150$ bytes);

\item
\textbf{\texttt{sum}}: SPARC executable file ($38240$ bytes).
\end{itemize}

We chose these files to have a wide span among different kinds of files and show, in this way, the several behaviours the algorithm can assume for different scenarios. In particular, we use the file \texttt{rep.txt} to obtain the maximum compression capability of the algorithm, the file \texttt{per} was included to show the behaviours in presence of periodicity and the file \texttt{ran} represents the worst scenario for compression purposes, since it has no redundancy to exploit and a high entropy. The last four files, on the other hand, do not represent borderline or special cases, but more common examples of files one can deals with during his ordinary life. in the next sections we will see how the algorithms behave working with each of these files.

\subsection{Redundant file}
The first file, \texttt{rep.txt}, is the most redundant file we can have: $10000$ repetitions of the same character, \texttt{A}. The nature of the file allows a very strong compression, since we could just say what is the symbol and how many times it is repeated. The following table shows the performances of the algorithms executed with a fixed \textit{coding window} of length $2000$ and a \textit{searching window} varying between $1000$, $5000$ and $10000$, which is the length of the message itself:
\begin{center}
\begin{tabular}{r | c | c |}
\multicolumn{3}{c |}{\texttt{rep.txt}} \\ \hline
$L_s$ & LZ77 & LZSS \\ \hline
1000 & $0.25$\% & $0.19$\% \\
5000& $0.27$\% & $0.20$\% \\
10000& $0.28$\% & $0.21$\% \\
\hline
\end{tabular}
\end{center}

As we expected the compression is very high: we can represent $10000$ bytes with less than $30$ bytes. It may look strange that performances drop with the increasing of the \textit{searching window}; this is due to the number of bits we need to encode a triplet, which raises with the length of the window (in LZ77 we use $29$ bits when $L_s = 1000$, $32$ bits when $L_s = 5000$ and $35$ bits when $L_s = 1000$). LZSS works to solve this issue, as a matter of fact it behaves slightly better than LZ77 and manages to encode the dictionary with less bits. In this case we can see that the length of the \textit{coding window} has also a great relevance on performances; the followig table summarizes the results obtained when we use the same three searching lengths of the previous example, but a coding length $L_c = 10000$:
\begin{center}
\begin{tabular}{r | c | c |}
\multicolumn{3}{c |}{\texttt{rep.txt}} \\ \hline
$L_s$ & LZ77 & LZSS \\ \hline
1000 & $0.11$\% & $0.08$\% \\
5000& $0.22$\% & $0.08$\% \\
10000& $0.12$\% & $0.09$\% \\
\hline
\end{tabular}
\end{center}
In this case we just need about ten bytes to represent the whole message and the dictionary is composed by only two rows.
It is interesting to report here the 3D plot of the LZ77 performances, in percentage of compression, in function of $L_s$ and $L_c$:

\begin{center}
\begin{figure}[H]
\includegraphics[width=8.5cm]{images/rep_surf.png}
\caption{LZ77 compression ratio in function of $L_s$ and $L_c$.}
\end{figure}
\end{center}

\subsection{Periodic file}
A big problem with implicit dictionary based compression algorithms is that periodicity of the message could not be properly exploited. This unfavorable event occurs if the windows are shorter than the period of the message, such that there is no way the algorithm can become aware of the existance of periodicity in the message. We can bring a clear example compressing the file \texttt{per} keeping fixed the \textit{coding window} to $2000$, but with two different \textit{searching window} lengths: $500$ (half a period) and $5000$ (five times a period). The compression results are reported in the following table:

\begin{center}
\begin{tabular}{r | c | c |}
\multicolumn{3}{c |}{\texttt{per}} \\ \hline
$L_s$ & LZ77 & LZSS \\ \hline
500 & $189.98$\% & $112.53$\% \\
5000& $23.55$\% & $11.44$\% \\
\hline
\end{tabular}
\end{center}

It is clear that, in the first case, the algorithm is not exploiting the periodicity, in fact it is expanding the data and increasing the file dimension, because, being the period built randomly, it finds no redundancy to exploit. In the second scenario, however, periodicity is exploited and the file is strongly compressed.

\subsection{Randomly generated file}
When a file is made by a random sequence of bytes its entropy is very high and the algorithm finds no redundancy to exploit for compression. In this case performances are very low, in fact the output file has a greater dimension than the input file and we can do very little also by varying $L_s$ and $L_c$. The following table reports the compression ratios with $L_c = 2000$ and $L_s = 1000, \ 5000, \ 10000$:

\begin{center}
\begin{tabular}{r | c | c |}
\multicolumn{3}{c |}{\texttt{ran}} \\ \hline
$L_s$ & LZ77 & LZSS \\ \hline
1000 & $184.66$\% & $112.53$\% \\
5000& $197.83$\% & $112.55$\% \\
10000& $202.12$\% & $112.56$\% \\
\hline
\end{tabular}
\end{center}

We can notice two main facts: the first is that LZSS compresses in general better than LZ77; the second is that LZSS' performances decrease much more slowly than LZ77's; if we inspect the dictionaries created by the two algorithms, we can try to spot the reason of such difference: the LZ77 dictionary contains a lot of references to an only previous symbol and we cannot avoid using a whole triplet to encode them. On the other hand the LZSS algorithm is able to realize that it is not a good deal to waste a whole triplet to express an only symbol and it encodes it as a single byte. In this way we ideally would have a dictionary with the same dimensions of the message, but, unlucklily, the coding algorithm also includes the flag bits. Thus, it is practically quite impossible to compress a random sequence: if we try to compress it, we are going to hopelessly expand it.

\subsection{Common files}
The file \texttt{shak.txt} is an example of human-written text file. As we can see from the results of the experiment, LZ77 and LZSS are not the best choice for compressing this kind of file. The reason is that the spoken language has not a strong redundancy and patterns of characters are often short and repeated only a few times. The table below shows the performances with $L_c = 2000$:
\begin{center}
\begin{tabular}{r | c | c |}
\multicolumn{3}{c|}{\texttt{shak.txt}} \\ 
\hline
$L_s$ & LZ77 & LZSS \\ \hline
1000 & $89.48$\% & $77.91$\% \\
5000& $82.26$\% & $97.27$\% \\
10000& $81.38$\% & $102.53$\% \\
\hline
\end{tabular}
\end{center}

Code files as \texttt{web.html} and \texttt{code.c} presents a greater redundancy because of their scientific format; strict rules of the programming languages make them very repetitive and thus the LZ77 and the LZSS manage to perform a better compression of such files. The following tables show the results for the two files where a $2000$ symbols long \textit{coding window} has been used:
\begin{center}
\begin{tabular}{r | c | c |}
\multicolumn{3}{c|}{\texttt{code.c}} \\
\hline
$L_s$ & LZ77 & LZSS \\ \hline
1000 & $73.53$\% & $59.73$\% \\
5000& $62.25$\% & $71.35$\% \\
10000& $58.68$\% & $76.14$\% \\
\hline
\end{tabular}

\vspace{0.5cm}

\begin{tabular}{r | c | c |}
\multicolumn{3}{c|}{\texttt{web.html}} \\
\hline
$L_s$ & LZ77 & LZSS \\ \hline
1000 & $59.82$\% & $49.27$\% \\
5000& $51.65$\% & $57.49$\% \\
10000& $51.45$\% & $60.19$\% \\
\hline
\end{tabular}
\end{center}

The executable file \texttt{sum} can also be quite well compressed:
\begin{center}
\begin{tabular}{r | c | c |}
\multicolumn{3}{c|}{\texttt{sum}} \\ 
\hline
$L_s$ & LZ77 & LZSS \\ \hline
1000 & $68.91$\% & $60.58$\% \\
5000& $61.85$\% & $68.69$\% \\
10000& $55.83$\% & $71.34$\% \\
\hline
\end{tabular}
\end{center}

From the last three tables it could seem that the LZ77 improves with the increasing of the length $L_s$, while the LZSS does the opposite. Actually things are a bit more complicated, as we can see by plotting a 3D graph of the compression ratio in function of both $L_s$ and $L_c$. We report here the plots for the files \texttt{shak.txt} and \texttt{code.c}, both for the LZ77 and LZSS. 

\begin{center}
\begin{figure}[H]
\includegraphics[width=8.5cm]{images/rep_surf.png}
\caption{LZ77 compression ratio for \texttt{shak.txt}.}
\end{figure}
\end{center}

\begin{center}
\begin{figure}[H]
\includegraphics[width=8.5cm]{images/rep_surf.png}
\caption{LZSS compression ratio for \texttt{shak.txt}.}
\end{figure}
\end{center}

\begin{center}
\begin{figure}[H]
\includegraphics[width=8.5cm]{images/rep_surf.png}
\caption{LZ77 compression ratio for \texttt{code.c}.}
\end{figure}
\end{center}

\begin{center}
\begin{figure}[H]
\includegraphics[width=8.5cm]{images/rep_surf.png}
\caption{LZSS compression ratio for \texttt{code.c}.}
\end{figure}
\end{center}

%It is easy to see that performances tends to improve with the growing of the \textit{searching window}. We could expect such a result: with shorter windows it is more difficult to find repeated pattern to refer to and, in some cases, we could neither realize that such patterns are present in the message. For a highly redundant input stream a long window allows to encode at the same time much more repetitions; even though the number of bits needed to represent a wide range of values grows with the increasing of the window, this happens with a logarithmic trend, while the amount of space used to encode a sequence byte per byte grows linearly. Clearly the usage of large windows implies a drop of performances from a computational time point of view. 
%In the best compression scenario, we would like to have a \textit{search window} as long as the message itself, but, since this is not suitable for time reason, commercial softwares usually work with block of symbols $32$ Kbyte long with windows of


\section{Comparison with existing softwares}
In this section we report some results obtained by the comparison between our implementations and other softwares usually installed in common PCs. We use the so called \textit{Canterbury Corpus} as set of files to perform our examination, which is a collection of several types of file whose principal aim is being used to try compression algorithms. I want to mention right now that our implementations have no chance to behave as well as any of the professional software employed by default by a PC. The reasons are mainly two: the first one is that programs that compress files or folders into archives .ZIP use an advanced algorithm, where the simple LZ77 is enhanced by further coding, for example by a Huffman coding (this combined coding is called \textit{LZ77 deflate} and is used by most of the .ZIP compressors). The second reason is about time: since we execute our program inside Matlab, performances are lowered by the infrastructure of the environment and, moreover, \texttt{for} cycles are not very convenient in this particular language. Hence, rather than observing performances in an absolute way, we will investigate the types of files which LZ77 and LZSS are favorable for and we will observe when the implicit dictionary-based coding are a suitable choice or not.


%\section{Results}
In this section we illustrate some considerable results we found during the experience. The first part of the section shows the different performances reached by using the \textit{LZ77} algorithm with different lengths of the \textit{searching window}, the second part focuses on the combined effect of changing both \textit{searching window} and \textit{coding window} and the last part compares our \textit{LZ77} and \textit{LZSS} implementations with commercial softwares.

All the experiments we performed have been executed on seven specific types of files, part of them taken from the \textit{Canterbury Corpus}, a collection of files with several formats to be used as sample files for compression testing. The used files are here reported (during the experimentation and the report itself we refer to each file by its index number):

\begin{itemize}
\item
\textbf{\texttt{file\_1}}: simple text file containing a sequence of character \texttt{A} repeated $10000$ times ($10000$ bytes);

\item
\textbf{\texttt{file\_2}}: simple text file containing a repetition of the latin alphabet ($13000$ bytes);

\item
\textbf{\texttt{file\_3}}: file generated by concatening $10000$ random bytes ($10000$ bytes);

\item
\textbf{\texttt{file\_4}}: simple text file reporting a piece of a Shakespeare poem ($13741$ bytes);

\item
\textbf{\texttt{file\_5}}: html format page ($24603$ bytes);

\item
\textbf{\texttt{file\_6}}: piece of \texttt{c} code ($11150$ bytes);

\item
\textbf{\texttt{file\_7}}: SPARC executable file ($38240$ bytes).
\end{itemize}

We chose these files to have a wide span among different kinds of files and show, in this way, the several behaviours the algorithm can assume for different scenarios. In particular, we use the file \texttt{file\_1} to obtain the maximum compression capability of the algorithm, the file \texttt{file\_2} was included to show the behaviours in presence of periodicity and the file \texttt{file\_3}, which, being randomly generated has no redundancy to exploit and a high entropy, represents the worst scenario for compression purposes. The last four files, on the other hand, do  not represent borderline or special cases, but more common examples of files one can deals with during his ordinary life.

\subsection{\textit{Searching window} variations}
A big problem with implicit dictionary based compression algorithms is that periodicity of the message could not be properly exploited and thus we could be wasting compression resources. This unfavorable event occurs if the \textit{searching window} is shorter than the period of the message, such that there is no way the algorithm can become aware of the existance of periodicity in the message. We ran the algorithm on the seven chosen files using different lengths for the \textit{searching window}. Hereafter we report the obtained curves:

%\begin{center}
%\begin{figure}[H]
%\includegraphics[width=\textwidth]{report_img/snr.eps}
%\caption{Comparison between systems performances.}
%\end{figure}
%\end{center}

It is easy to see that performances tends to improve with the growing of the \textit{searching window}. We could expect such a result: with shorter windows it is more difficult to find repeated pattern to refer to and, in some cases, we could neither realize that such patterns are present in the message. For a highly redundant input stream a long window allows to encode at the same time much more repetitions; even though the number of bits needed to represent a wide range of values grows with the increasing of the window, this happens with a logarithmic trend, while the amount of space used to encode a sequence byte per byte grows linearly. Clearly the usage of large windows implies a drop of performances from a computational time point of view. 
In the best compression scenario, we would like to have a \textit{search window} as long as the message itself, but, since this is not suitable for time reason, commercial softwares usually work with block of symbols $32$ Kbyte long with windows of

\subsection{\textit{Searching window} and \textit{coding window} variations}
The \textit{LZ77} algorithm has, in fact, two parameters we can act to: the lengths of the \textit{searching window} and of the \textit{coding window}. It could be interesting wondering what happens if we stretch both the windows. From the following 2D graphs we can see that the length of the \textit{coding window} has, actually, a low impact on coding performances. The reason is that, for a redundant message, 

\bibliographystyle{plain}
\bibliography{src}

\end{document}